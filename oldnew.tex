
\documentclass[10pt]{article}
\usepackage[left=1.5cm,top=1.5cm,hcentering,a4paper]{geometry}
\usepackage{siunitx}
\usepackage{url}
\usepackage{paracol}

\begin{document}

\setcounter{secnumdepth}{0}

\title{Physics national curriculum key stage 3\\2014 cf.\ 1999}
\author{A.C.\ Norman}
\date{June 2014}
\maketitle

\thispagestyle{empty}

\setlength{\parindent}{0pt}

\begin{paracol}{2}

\begin{center}
\Large 1999
\end{center}

\switchcolumn

\begin{center}
\Large 2014
\end{center}

\switchcolumn

\switchcolumn

Pupils should be taught about:
\section{Energy}
\subsection{Calculation of fuel uses and costs in the domestic context}
\begin{itemize}
\item comparing energy values of different foods\\ (from labels) (kJ)
\item comparing power ratings of appliances in watts\\ (W, kW)
\item comparing amounts of energy transferred\\ (J, kJ, kW hour)
\item domestic fuel bills, fuel use and costs
\item fuels and energy resources.
\end{itemize}
\subsection{Energy changes and transfers}
\begin{itemize}
\item  simple machines give bigger force but at the expense of smaller movement (and vice
versa): product of force and displacement unchanged
\item heating and thermal equilibrium: temperature difference between two objects leading to
energy transfer from the hotter to the cooler one, through contact (conduction) or
radiation; such transfers tending to reduce the temperature difference: use of insulators
\item other processes that involve energy transfer: changing motion, dropping an object,
completing an electrical circuit, stretching a spring, metabolism of food, burning fuels.
\subsection{Changes in systems}
\item energy as a quantity that can be quantified and calculated; the total energy has the
same value before and after a change
\item comparing the starting with the final conditions of a system and describing increases
and decreases in the amounts of energy associated with movements, temperatures,
changes in positions in a field, in elastic distortions and in chemical compositions
\item using physical processes and mechanisms, rather than energy, to explain the
intermediate steps that bring about such changes.
\end{itemize}

\switchcolumn

{}
\section{Energy resources and\\energy transfer}
\begin{enumerate}
\setcounter{enumi}{4}
\item Pupils should be taught:
\subsection{Energy resources}
\begin{enumerate}
\item about the variety of energy resources, including oil, gas, coal, biomass,
food, wind, waves and batteries, and the distinction between renewable 
and non-renewable resources
\item about the Sun as the ultimate source of most of the Earth's energy resources
and to relate this to how coal, oil and gas are formed
\item that electricity is generated by means of a variety of energy resources
\subsection{Conservation of energy}
\item the distinction between temperature and heat, and that differences 
in temperature can lead to transfer of energy
\item ways in which energy can be usefully transferred and stored
\item how energy is transferred by the movement of particles in conduction,
convection and evaporation, and that energy is transferred directly 
by radiation
\item that although energy is always conserved, it may be dissipated, reducing 
its availability as a resource.
\end{enumerate}
\end{enumerate}

\switchcolumn*

\section{Motion and forces}
\subsection{Describing motion}
\begin{itemize}
\item speed and the quantitative relationship between average speed, distance and time\\
(speed = distance $\div$ time)
\item the representation of a journey on a distance-time graph
\item relative motion: trains and cars passing one another.
\end{itemize}

\subsection{Forces}
\begin{itemize}
\item forces as pushes or pulls, arising from the interaction between two objects
\item using force arrows in diagrams, adding forces in one dimension, balanced and
unbalanced forces
\item moment as the turning effect of a force
\item forces: associated with deforming objects; stretching and squashing --- springs; with
rubbing and friction between surfaces, with pushing things out of the way; resistance to
motion of air and water
\item forces measured in newtons, measurements of stretch or compression as force is
changed
\item force-extension linear relation; Hooke's Law as a special case
\item work done and energy changes on deformation
\item non-contact forces: gravity forces acting at a distance on Earth and in space, forces
between magnets and forces due to static electricity.
\end{itemize}

\subsection{Pressure in fluids}
\begin{itemize}
\item atmospheric pressure, decreases with increase of height as weight of air above
decreases with height
\item pressure in liquids, increasing with depth; upthrust effects, floating and sinking
\item pressure measured by ratio of force over area --- acting normal to any surface.
\end{itemize}

\subsection{Balanced forces}
\begin{itemize}
\item opposing forces and equilibrium: weight held by stretched spring or supported on a
compressed surface.
\end{itemize}

\subsection{Forces and motion}
\begin{itemize}
\item forces being needed to cause objects to stop or start moving, or to change their speed
or direction of motion (qualitative only)
\item change depending on direction of force and its size.
\end{itemize}

\switchcolumn

\section{Forces and motion}
\begin{enumerate}
\setcounter{enumi}{1}
\item Pupils should be taught:
\begin{enumerate}
\subsection{Force and linear motion}
\item how to determine the speed of a moving object and to use the quantitative
relationship between speed, distance and time
\item that the weight of an object on Earth is the result of the gravitational
attraction between its mass and that of the Earth
\item that unbalanced forces change the speed or direction of movement of
objects and that balanced forces produce no change in the movement 
of an object
\item ways in which frictional forces, including air resistance, affect motion 
[for example, streamlining cars, friction between tyre and road]
\subsection{Force and rotation}
\item that forces can cause objects to turn about a pivot
\item the principle of moments and its application to situations involving one pivot
\subsection{Force and pressure}
\item the quantitative relationship between force, area and pressure and its
application [for example, the use of skis and snowboards, the effect of sharp
blades, hydraulic brakes].
\end{enumerate}
\end{enumerate}

\switchcolumn*

\section{Waves}
\subsection{Observed waves}
\begin{itemize}
\item waves on water as undulations which travel through water with transverse motion;
these waves can be reflected, and add or cancel --- superposition.
\end{itemize}

\subsection{Sound waves}
\begin{itemize}
\item frequencies of sound waves, measured in hertz (Hz); echoes, reflection and absorption
of sound
\item sound needs a medium to travel, the speed of sound in air, in water, in solids
\item sound produced by vibrations of objects, in loud speakers, detected by their effects on
microphone diaphragm and the ear drum; sound waves are longitudinal
\item auditory range of humans and animals.
\end{itemize}

\subsection{Energy and waves}
\begin{itemize}
\item pressure waves transferring energy; use for cleaning and physiotherapy by ultra-sound;
waves transferring information for conversion to electrical signals by microphone.
\end{itemize}

\subsection{Light waves}
\begin{itemize}
\item the similarities and differences between light waves and waves in matter
\item light waves travelling through a vacuum; speed of light
\item the transmission of light through materials: absorption, diffuse scattering and specular
reflection at a surface
\item use of ray model to explain imaging in mirrors, the pinhole camera, the refraction of
light and action of convex lens in focusing (qualitative); the human eye
\item light transferring energy from source to absorber leading to chemical and electrical
effects; photo-sensitive material in the retina and in cameras
\item colours and the different frequencies of light, white light and prisms (qualitative only);
differential colour effects in absorption and diffuse reflection.
\end{itemize}

\switchcolumn

\section{Light and sound}
\begin{enumerate}
\setcounter{enumi}{2}
\item Pupils should be taught:
\begin{enumerate}
\subsection{The behaviour of light}
\item that light travels in a straight line at a finite speed in a uniform medium
\item that non-luminous objects are seen because light scattered from them 
enters the eye
\item how light is reflected at plane surfaces
\item how light is refracted at the boundary between two different materials
\item that white light can be dispersed to give a range of colours
\item the effect of colour filters on white light and how coloured objects appear 
in white light and in other colours of light
\subsection{Hearing}
\item that sound causes the eardrum to vibrate and that different people have
different audible ranges
\item some effects of loud sounds on the ear [for example, temporary deafness]
\subsection{Vibration and sound}
\item that light can travel through a vacuum but sound cannot, and that light
travels much faster than sound
\item the relationship between the loudness of a sound and the amplitude 
of the vibration causing it
\item the relationship between the pitch of a sound and the frequency 
of the vibration causing it.
\end{enumerate}
\end{enumerate}

\switchcolumn*

\section{Electricity and electromagnetism}
\subsection{Current electricity}
\begin{itemize}
\item electric current, measured in amperes, in circuits, series and parallel circuits, currents
add where branches meet and current as flow of charge
\item potential difference, measured in volts, battery and bulb ratings; resistance, measured
in ohms, as the ratio of potential difference (p.d.) to current
\item differences in resistance between conducting and insulating components (quantitative).
\end{itemize}
\subsection{Static electricity}
\begin{itemize}
\item separation of positive or negative charges when objects are rubbed together: transfer
of electrons, forces between charged objects
\item the idea of electric field, forces acting across the space between objects not in contact.
\end{itemize}
\subsection{Magnetism}
\begin{itemize}
\item magnetic poles, attraction and repulsion
\item magnetic fields by plotting with compass, representation by field lines
\item Earth's magnetism, compass and navigation
\item the magnetic effect of a current, electromagnets, D.C. motors (principles only).
\end{itemize}

\switchcolumn

\section{Electricity and magnetism}
\begin{enumerate}
\setcounter{enumi}{0}
\item Pupils should be taught:
\begin{enumerate}
\subsection{Circuits}
\item how to design and construct series and parallel circuits, and how to measure
current and voltage
\item that the current in a series circuit depends on the number of cells and the
number and nature of other components and that current is not `used up'
by components
\item that energy is transferred from batteries and other sources to other
components in electrical circuits
\subsection{Magnetic fields}
\item about magnetic fields as regions of space where magnetic materials
experience forces, and that like magnetic poles repel and unlike poles attract
\subsection{Electromagnets}
\item that a current in a coil produces a magnetic field pattern similar to that 
of a bar magnet
\item how electromagnets are constructed and used in devices [for example, relays,
lifting magnets].
\end{enumerate}
\end{enumerate}

\switchcolumn*

\section{Matter}
\subsection{Physical changes}
\begin{itemize}
\item conservation of material and of mass, and reversibility, in melting, freezing,
evaporation, sublimation, condensation, dissolving
\item similarities and differences, including density differences, between solids, liquids and
gases
\item Brownian motion in gases
\item diffusion in liquids and gases driven by differences in concentration
\item the difference between chemical and physical changes.
\end{itemize}
\subsection{Particle model}
\begin{itemize}
\item the differences in arrangements, in motion and in closeness of particles explaining
changes of state, shape and density, the anomaly of ice-water transition
\item atoms and molecules as particles.
\end{itemize}
\subsection{Energy in matter}
\begin{itemize}
\item changes with temperature in motion and spacing of particles
\item internal energy stored in materials.
\end{itemize}

\switchcolumn

\fbox{
  \parbox{0.45\textwidth}{
Corresponding 1994 content for the `Matter' content in the 2014 curriculum is found in Sc3 (`materials and their properties') rather than Sc4 (`physical processes').
  }
}

\begin{enumerate}
\section{Changing materials}
\setcounter{enumi}{1}
\item Pupils should be taught:
\begin{enumerate}
\subsection{Physical changes}
\item that when physical changes [for example, changes of state, formation
of solutions] take place, mass is conserved
\item about the variation of solubility with temperature, the formation of saturated
solutions, and the differences in solubility of solutes in different solvents
\item to relate changes of state to energy transfers
\end{enumerate}
\section{Classifying materials}
\setcounter{enumi}{0}
\item Pupils should be taught:
\begin{enumerate}
\subsection{Solids, liquids and gases}
\item how materials can be characterised by melting point, boiling point
and density
\item how the particle theory of matter can be used to explain the properties
of solids, liquids and gases, including changes of state, gas pressure
and diffusion
\end{enumerate}
\end{enumerate}

\switchcolumn*

\section{Space physics}
\begin{itemize}
\item gravity force, $\text{weight} = \text{mass}\times\text{gravitational field strength (g)}$, on Earth $g=\SI{10}{N/kg}$,
different on other planets and stars; gravity forces between Earth and Moon, and
between Earth and Sun (qualitative only)
\item our Sun as a star, other stars in our galaxy, other galaxies
\item the seasons and the Earth's tilt, day length at different times of year, in different
hemispheres
\item the light year as a unit of astronomical distance.
\end{itemize}

\switchcolumn

\section{The Earth and beyond}
\begin{enumerate}
\setcounter{enumi}{3}
\item Pupils should be taught:
\begin{enumerate}
\subsection{The solar system}
\item how the movement of the Earth causes the apparent daily and annual
movement of the Sun and other stars
\item the relative positions of the Earth, Sun and planets in the solar system
\item about the movements of planets around the Sun and to relate these 
to gravitational forces
\item that the Sun and other stars are light sources and that the planets and 
other bodies are seen by reflected light
\item about the use of artificial satellites and probes to observe the Earth 
and to explore the solar system.
\end{enumerate}
\end{enumerate}

\end{paracol}

\vfill

\begin{minipage}[t]{0.45\textwidth}
\footnotesize\raggedright
\copyright\ Crown copyright 2004\\
This material is reproduced under the terms of the copyright guidance issued by HMSO\\
\url{www.hmso.gov.uk/guides.htm}\\
\end{minipage}
\hfill
\begin{minipage}[t]{0.45\textwidth}
\footnotesize
\copyright\ Crown copyright 2013\\
This information is used under the terms of the Open Government Licence\\
\url{http://www.nationalarchives.gov.uk/doc/open-government-licence/}\\
\end{minipage}


\end{document}
