
\documentclass[a4paper,12pt]{article}
\usepackage{acn-questions}
\usepackage{url}

\begin{document}

\setcounter{secnumdepth}{0}

\title{Science national curriculum key stage 3\\Forces}
\author{Department for Education}
\date{September 2013}
\maketitle

\thispagestyle{empty}

Pupils should be taught about:

\begin{itemize}
\item forces as pushes or pulls, arising from the interaction between two objects
\item using force arrows in diagrams, adding forces in one dimension, balanced and
unbalanced forces
\item moment as the turning effect of a force
\item forces: associated with deforming objects; stretching and squashing --- springs; with
rubbing and friction between surfaces, with pushing things out of the way; resistance to
motion of air and water
\item forces measured in newtons, measurements of stretch or compression as force is
changed
\item force-extension linear relation; Hooke's Law as a special case
\item work done and energy changes on deformation
\item non-contact forces: gravity forces acting at a distance on Earth and in space, forces
between magnets and forces due to static electricity.
\end{itemize}

\subsection{Balanced forces}
\begin{itemize}
\item opposing forces and equilibrium: weight held by stretched spring or supported on a
compressed surface.
\end{itemize}

\vfill

\footnotesize
\noindent \copyright\ Crown copyright 2013\\
This information is used under the terms of the Open Government Licence\\
\url{http://www.nationalarchives.gov.uk/doc/open-government-licence/}\\
{\tiny Reference: DFE-00185-2013}

\end{document}
