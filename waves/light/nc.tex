
\documentclass[a4paper,12pt]{article}
\usepackage{acn-questions}
\usepackage{url}

\begin{document}

\setcounter{secnumdepth}{0}

\title{Science national curriculum key stage 3\\Light waves}
\author{Department for Education}
\date{September 2013}
\maketitle

\thispagestyle{empty}

Pupils should be taught about:

\subsection{Light waves}
\begin{itemize}
\item the similarities and differences between light waves and waves in matter
\item light waves travelling through a vacuum; speed of light
\item the transmission of light through materials: absorption, diffuse scattering and specular
reflection at a surface
\item use of ray model to explain imaging in mirrors, the pinhole camera, the refraction of
light and action of convex lens in focusing (qualitative); the human eye
\item light transferring energy from source to absorber leading to chemical and electrical
effects; photo-sensitive material in the retina and in cameras
\item colours and the different frequencies of light, white light and prisms (qualitative only);
differential colour effects in absorption and diffuse reflection.
\end{itemize}
\subsection{Observed waves}
\begin{itemize}
\item waves on water as undulations which travel through water with transverse motion;
these waves can be reflected, and add or cancel --- superposition.
\end{itemize}

\vfill

\footnotesize
\noindent \copyright\ Crown copyright 2013\\
This information is used under the terms of the Open Government Licence\\
\url{http://www.nationalarchives.gov.uk/doc/open-government-licence/}\\
{\tiny Reference: DFE-00185-2013}

\end{document}
